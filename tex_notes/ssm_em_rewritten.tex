\documentclass[10pt]{article}
\usepackage[margin=1in]{geometry} 
\usepackage{amsmath,amsthm,amssymb,amsfonts}
\usepackage{algorithm}
\usepackage{algpseudocode}
 
\DeclareMathOperator*{\argmax}{arg\,max}
\newcommand\nlf{n_{\rm LF}}
 
\begin{document}
 
\title{EM for State Space Models with Missing Data}
\author{Adam Coogan}
\maketitle

\section{Model}

The state space model (SSM) has the form
\begin{align}
    \vec{x}_t &= A_t \vec{x}_{t-1} + B_t \vec{u}_t + \vec{\varepsilon}_t\\
    \vec{y}_t &= C_t \vec{x}_t + D_t \vec{v}_t + \vec{\delta}_t,
\end{align}
where $\vec{u}_t$ and $\vec{v}_t$ are the state transition and observation controls and $\vec{\varepsilon}_t \sim N(0, Q_t)$ and $\vec{\delta}_t \sim N(0, R_t)$. The dimensions of the vectors $\vec{x}_t$ ,$\vec{y}_t$, $\vec{u}_t$, $\vec{v}_t$ are $\nlf$, $N$, $L$ and $M$. $T$ denotes the number of observations and $\mathcal{D}$ the set of obserations.

In our scenario, $N > \nlf$, and the hidden state $\vec{x}_t$ does not have an obvious physical interpretation. This means we need to learn the parameters of the model $\theta = {A, B, C, D, Q, R}$. To improve numerical stability, I set $Q = I$ and take $R$ to be diagonal.

\end{document}


